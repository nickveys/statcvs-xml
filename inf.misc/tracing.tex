\section[Tracing]{Project Tracing}
\frame{
    \frametitle{Verfolgung von Projekten}
    Tracing ist bestandteil der Dokumentation von Projekten. Prozessdoku.
    OpenSource: Erfolgreiche OS-Projekte brauchen 4 Dinge: Projekt, Code, Dokumentation, Community.
    Apache Way of Life.
    H�ufigster Fehler: keine Doku.
    Lehrbuchmodell: schreibt die Dokumentation schon vor.
    Dokumentation ist aufwendig, deshalb automatisieren.
}

\frame{
    \frametitle{Erfolgreiche Projekte}
    Erfolgreiche Projekte brauchen 4 Dinge:
    \begin{itemize}
        \item das Projekt (bildet den Rahmen)
        \item den Quellcode (klar!)
        \item \alert<2>{ausf�hrliche Dokumentation (!!)}
        \item eine Community / einen Kunden.
    \end{itemize}
    \vspace{0.5cm}
    \invisible<1>{Zur Dokumentation geh�ren
    \begin{itemize}
        \item Beschreibung der Schnittstellen (API Dokumentation)
        \item Ergebnisse der Unit-Tests inkl. Messung der Test�berdeckung
        \item Erhebung und Visualisierung von Metriken
        \item Produktdokumentation (Handb�cher, Online-Hilfe etc.)
    \end{itemize}}
}

\frame{
    \frametitle{Automatisierter Buildprozess}
    Autmatisches Build gr��erer Projekte mit Ant und Make.
    Nachteile:
    \begin{itemize}
        \item{Targets k�nnen nicht zwischen Projekten geteilt werden}
        \item{Builddateien fast identisch}
        \item{Kein Scripting m�glich (keine Loops/Conditionals)}
        \item{Verwendung von 3rd-Party-Tools (z.B. javadoc) relativ umst�ndlich}
        \item{Keine Aufl�sung von Library-Dependencies (Jar-H�lle)}
    \end{itemize}
    \vspace{0.5cm}
    \large{\bf{L�sung:} Apache Maven}
}

