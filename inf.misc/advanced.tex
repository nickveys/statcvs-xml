\section{Bootstrap}

\subsection{Bootstrap}
\frame[all:1]{
  \frametitle{Bootstrap}
  
  Maven ist sein eigenes Build Tool.
}


\section{Tipps}

\subsection{Command Line}
\frame[all:1]{
  \frametitle{Command Line Optionen}
  
  \begin{itemize}
  \item \begin{alltt}-g\end{alltt} Eine Liste aller Goals anzeigen
  \item -o Offline Modus. Nur Goals ausf�hren, die keine Internet
    Verbindung ben�tigen
  \item -p Project Datei angeben. N�tzlich f�r aufrufe aus nicht Projekt
    Verzeichnissen (z.B. in einem cron job).
  \end{itemize}
}

\subsection{Konfiguration}
\frame[all:1]{
  \frametitle{Konfiguration}
  
  Pers�nliche Einstellungen k�nnen in \$HOME/build.properties
  vorgenommen werden.

  \begin{alltt}
    maven.repo.local=/usr/local/maven-1.0-rc1/repository
    maven.home.local=/usr/local/maven-1.0-rc1
    java.compiler=modern
    maven.username=nick
  \end{alltt}
}

\subsection{Customizing}
\frame[all:1]{
  \frametitle{Customizing}
  
  \begin{itemize}
  \item Jelly: Code und Layout bunt durcheinander gemischt
  \item xnap.jsl Ein Beispiel f�r ein 3 Column Layout
  \item Includieren von php Skripten 
  \end{itemize}

\begin{alltt}
<jsl:template match="include" trim="false">
  <x:set var="_file" select="string(@file)"/>
  <jsl:comment>#include virtual="${_file}"</jsl:comment>
</jsl:template>
\end{alltt}

\begin{alltt}
<section name="FAQ">
<p><include file="faq.php"/></p>
\end{alltt}

}

%%% Local Variables: 
%%% TeX-master: "index"
%%% End: 
